%%%%%%%%%%%%%%%%%%%%%%%%%%%%%%%%%%%%%%%%%%%%%%%%%%%%%%%%%%%%%%%%%%%%%%%%%%%%%%%%
%%% Clusterização de Dados Biomédicos com Técnicas de Algoritmos Genéticos   %%%
%%%%%%%%%%%%%%%%%%%%%%%%%%%%%%%%%%%%%%%%%%%%%%%%%%%%%%%%%%%%%%%%%%%%%%%%%%%%%%%%

Esse projeto apresentou a parte inicial de um estudo novo sobre aplicação de algorítmos genéticos e técnicas de {\it clustering} para dados biomédicos. Essa etapa de aprendizado pela falta de familiaridade com a área foi responsável por estabelecer conceitos sólidos que serão fundamentais para as próximas etapas.

Os algorítmos genéticos se mostraram uma ferramenta poderosa para os problemas de otimização e busca heurística, e podem melhorar o desempenho das técnicas de {\it clustering}. O custo computacional aumenta com a complexidade e tamanho das bases de dados. Portanto, os estudos na hibridização terão continuidade para o aprimoramento dessas técnicas e para atender aos objetivos estabelecidos para o projeto como um todo.

%%%%%%%%%%%%%%%%%%%%%%%%%%%%%%%%%%%%%%%%%%%%%%%%%%%%%%%%%%%%%%%%%%%%%%%%%%%%%%%%
%%%%%%%%%%%%%%%%%%%%%%%%%%%%%%%%%%%%%%%%%%%%%%%%%%%%%%%%%%%%%%%%%%%%%%%%%%%%%%%%
\section{Trabalhos Futuros}
Para a próxima etapa, as técnicas aprendidas serão utilizadas no desenvolvimento de novos modelos a partir da base dados. Será utilizada a abordagem CRISP-DM para a mineração de dados de maneira mais sistemática. As funções de aleatoriedade utilizadas até então serão substituídas pelo algorítmo LFSR implementado. E, também, técnicas mais rigorosas {\it cross validation} validarão os próximos modelos.

Os próximos algorítmos serão usados na própria base dados, para que já possam ser extraídas informações para que sejam comparadas futuramente.

Será necessário um estudo do domínio de aplicação, para o entendimento mínimo necessário para a interpretação correta das informações extraídas da base de dados.

É necessário que eficiência e eficácia dos algorítmos sejam atingidas, assim como resultados confiáveis e modelos robustos.

Com a finalização do projeto, apresentar os resultados no Congresso Anual de Iniciação Científica e submeter artigos à congressos internacionais.

O mais desejado é que as técnicas e resultados obtidos possam ser úteis à população, encontrando novos meios de prevenção e tratamento para as doenças causadas pelos microorganismos estudados. E que a continuação desse projeto possa ser incentivada.

%%%%%%%%%%%%%%%%%%%%%%%%%%%%%%%%%%%%%%%%%%%%%%%%%%%%%%%%%%%%%%%%%%%%%%%%%%%%%%%%
%%%%%%%%%%%%%%%%%%%%%%%%%%%%%%%%%%%%%%%%%%%%%%%%%%%%%%%%%%%%%%%%%%%%%%%%%%%%%%%%
\section{Cronograma para o próximo semestre}
Planejamento das atividades por mês:

• Meses 1, 2 e 3: Estudo do domínio de aplicação, proposição, construção e avaliação de técnicas baseados em algoritmos genéticos para a identificação de {\it clusters} na base de dados. 

• Meses 3 e 4 - Gerar documentação sobre o desenvolvimento da pesquisa, a ser defendida perante uma banca de pesquisadores em Inteligência Artificial dos Departamentos de Ciência da Computação da UnB e da Universidade de Tsukuba, Japão.

• Meses 5 e 6 - Preparar o relatório final, resumo e pôster do projeto visando à participação no Congresso Anual de Iniciação Científica e submeter artigos à congressos internacionais.