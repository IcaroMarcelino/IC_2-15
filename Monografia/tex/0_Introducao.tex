%%%%%%%%%%%%%%%%%%%%%%%%%%%%%%%%%%%%%%%%%%%%%%%%%%%%%%%%%%%%%%%%%%%%%%%%%%%%%%%%%%%%%%%%
%%%%%%%%%%%%%%%%%%%%%%%%               Introdução              %%%%%%%%%%%%%%%%%%%%%%%%%
%%%%%%%%%%%%%%%%%%%%%%%%%%%%%%%%%%%%%%%%%%%%%%%%%%%%%%%%%%%%%%%%%%%%%%%%%%%%%%%%%%%%%%%%
Nesse capítulo é elaborada uma descrição geral do projeto, definindo-o, apresentando
seus objetivos e algumas considerações sobre a proveniência dos dados que serão 
utilizados. Esse projeto será desenvolvido em parceria com o Prof. Dr. Claus Aranha da 
Universidade de Tsukuba, Japão.

%%%%%%%%%%%%%%%%%%%%%%%%%%%%%%%%%%%%%%%%%%%%%%%%%%%%%%%%%%%%%%%%%%%%%%%%%%%%%%%%%%%%%%%%
%%%%%%%%%%%%%%%%%%%%%%%%%%%%%%%%%%%%%%%%%%%%%%%%%%%%%%%%%%%%%%%%%%%%%%%%%%%%%%%%%%%%%%%%
\section{Definição do Problema}%
Medicina preventiva é uma área da saúde que objetiva prevenir a ocorrência de doenças, 
assim evitando gastos com tratamentos e melhorando a qualidade de vida do paciente. É 
de suma importância, principalmente em regiões menos afortunadas que ainda possuem 
problemas de infraestrutura e consequentemente saneamento básico deficiente, aumentando 
a incidência de doenças, principalmente infectocontagiosas.

Ainda hoje é difícil identificar muitas doenças, mesmo com sintomas iniciais. Já que um 
único sintoma é comum a muitos problemas. Algumas vezes, nem o próprio paciente 
consegue notar todas as anormalidades que estão ocorrendo no seu corpo. Os exames de 
rotina são extremamente importantes nesse aspecto. Porém, pequenas perturbações nas 
informações coletadas podem passar despercebidas.

Cada vez mais se têm novas informações sobre problemas de saúde, sejam sobre suas 
causas ou efeitos. Essa massa de informação se torna mais densa, logo, mais difícil de 
ser interpretada. Isso se torna um incentivo para uma análise superficial ou parcial, 
que pode esconder informações de muita relevância.

No tratamento de doenças causadas por microorganismos são usados antibióticos - compostos 
naturais ou sintéticos capazes de inibir o crescimento ou causar a morte de fungos ou 
bactérias \cite{antibioticos}. Porém seu uso constante pode trazer consequências piores
que os efeitos dos microorganismos.

A falta de conhecimento do medicamento correto para determinada situação, a falta de 
informação sobre as características do paciente são os principais fatores envolvidos 
na ocorrência das reações indesejadas pelo medicamento \cite{antibioticos1}.

Ferramentas computacionais já se mostraram um auxílio poderoso em diversos tratamentos 
médicos, por sua precisão e confiabilidade. De modo que, utilizando-as, é possível 
gerar resultados potencialmente mais precisos e rápidos que técnicas anteriores. 

Através de mineração de dados (do inglês, data mining - DM), é possível agrupar dados 
seguindo suas características, diferenciando-os e, ainda, mensurando sua 
confiabilidade. A proposta para esse projeto é a utilização de algumas técnicas de 
inteligência artificial e mineração de dados (serão definidas no capítulo 2) para, a 
partir da base de dados com informações sobre os microorganismos presentes nos 
pacientes, obter padrões sobre a efetividade de antibióticos e possíveis tendências 
para a ocorrência de um problema decorrente do microorganismo.

Como dito, existem muitas informações, exigindo um esforço computacional muito grande 
para seu processamento. Por isso, serão utilizados algoritmos genéticos para o 
aprimoramento de técnicas tradicionais de mineração de dados e o desenvolvimento de 
novas, de modo que sejam obtidas soluções mais rápidas e não piores que as anteriores.
%%%%%%%%%%%%%%%%%%%%%%%%%%%%%%%%%%%%%%%%%%%%%%%%%%%%%%%%%%%%%%%%%%%%%%%%%%%%%%%%%%%%%%%%
%%%%%%%%%%%%%%%%%%%%%%%%%%%%%%%%%%%%%%%%%%%%%%%%%%%%%%%%%%%%%%%%%%%%%%%%%%%%%%%%%%%%%%%%
\section{Sobre os dados que serão utilizados}%
A base de dados consiste em informações sobre culturas de micro-organismos encontradas 
em amostras coletadas de pacientes de diferentes faixas etárias de todo Brasil. Também 
apresenta características específicas dos micro-organismos e os antibióticos utilizados 
para o tratamento de cada um. Os dados provenientes do Ministério da Saúde são de 
domínio público. São 863834 amostras com 16 atributos associados (Será melhor descrito 
no capítulo 4) sobre os microorganismos e os pacientes. 

Vale ressaltar que os dados não possuem a identificação dos pacientes, não sendo necessário
obter termo de consentimento livre e esclarecido ou submissão à comissão de ética.
%%%%%%%%%%%%%%%%%%%%%%%%%%%%%%%%%%%%%%%%%%%%%%%%%%%%%%%%%%%%%%%%%%%%%%%%%%%%%%%%%%%%%%%%
%%%%%%%%%%%%%%%%%%%%%%%%%%%%%%%%%%%%%%%%%%%%%%%%%%%%%%%%%%%%%%%%%%%%%%%%%%%%%%%%%%%%%%%%
\section{Objetivos}
Esse projeto visa à obtenção de modelos relevantes quanto à eficiência de antibióticos
em microorganismos dado suas informações em pacientes de diferentes caracteríscas, 
sobre as bases de dados disponíveis. A eficiência e eficácia das técnicas utilizadas 
são de suma importância para o desenvolvimento, já que  são esperados modelos e 
resultados mais rápidos, robustos e confiáveis. 

Essa parte inicial do desenvolvimento consiste em um aprendizado dos conceitos 
fundamentais que serão as bases desse projeto e as discussões sobre as futuras 
aplicações.
%%%%%%%%%%%%%%%%%%%%%%%%%%%%%%%%%%%%%%%%%%%%%%%%%%%%%%%%%%%%%%%%%%%%%%%%%%%%%%%%%%%%%%%%
%%%%%%%%%%%%%%%%%%%%%%%%%%%%%%%%%%%%%%%%%%%%%%%%%%%%%%%%%%%%%%%%%%%%%%%%%%%%%%%%%%%%%%%%