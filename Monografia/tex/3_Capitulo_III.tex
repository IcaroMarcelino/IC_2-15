%%%%%%%%%%%%%%%%%%%%%%%%%%%%%%%%%%%%%%%%%%%%%%%%%%%%%%%%%%%%%%%%%%%%%%%%%%%%%%%%%%%%%%%%
%%%%%%%%%%%%%%%%%%%%%%%           Dados Utilizados           %%%%%%%%%%%%%%%%%%%%%%%%%%%
%%%%%%%%%%%%%%%%%%%%%%%%%%%%%%%%%%%%%%%%%%%%%%%%%%%%%%%%%%%%%%%%%%%%%%%%%%%%%%%%%%%%%%%%
Para testes nos algorítmos, foram selecionados dois bancos de dados conehcidos
mais simples e menores que os dados sobre microorganismos.

\section{Iris}
O banco de dados Iris \cite{uci} consiste em informações sobre três espécies 
diferentes de flor de íris. Contém 150 amostras cada uma com medidas do 
comprimento e largura médias de suas sépalas e comprimento e largura médias de 
suas pétalas, em centímetros. Foi o primeiro banco utilizado por ser muito pequeno
de modo a facilitar a visualização do funcionamento dos algorítmos.

% \section{Digitos manuscritos}
% Esse banco \cite{uci} foi criado coletando 250 amostras de 44 escritores, 
% destas 30 são usadas para treino e as outras 14 para treino independente. No total,
% são 10992 amostras com 16 atributos cada uma.


\section{Dados sobre microorganismos}
A base proveniente do Ministério da Saúde é composta por 863834 dados, com 16 atributos cada uma. Foram recolhidas de homens e mulheres, de todo o Brasil entre 0 e 100 anos de idade. O material biológico foi colhido entre 2008 e 2015 e têm grande variedade.

Os microorganismos são descritos por sua classe, família, gênero e espécie. É citada sua característica fenotípica e morfo-tintorial - padrão de coloração apresentado em sua identificação -, o antibiótico utilizado e se apresentou ou não resistência.