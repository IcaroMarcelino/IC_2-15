%%%%%%%%%%%%%%%%%%%%%%%%%%%%%%%%%%%%%%%%%%%%%%%%%%%%%%%%%%%%%%%%%%%%%%%%%%%%%%%%
%%%%%%%%%%%%%%%%%%%%              Aprendizado              %%%%%%%%%%%%%%%%%%%%%
%%%%%%%%%%%%%%%%%%%%%%%%%%%%%%%%%%%%%%%%%%%%%%%%%%%%%%%%%%%%%%%%%%%%%%%%%%%%%%%%
Nesse capítulo são apresentadas as ferramentas que serão utilizadas para o
desenvolvimento do projeto e a descrição dos primeiros algorítmos implementados.

%%%%%%%%%%%%%%%%%%%%%%%%%%%%%%%%%%%%%%%%%%%%%%%%%%%%%%%%%%%%%%%%%%%%%%%%%%%%%%%%
%%%%%%%%%%%%%%%%%%%%%%%%%%%%%%%%%%%%%%%%%%%%%%%%%%%%%%%%%%%%%%%%%%%%%%%%%%%%%%%%
\section{Linguagem Python}
Os algorítmos foram implementados na linguagem {\it Python} 3.4.3. Foi escolhida
pelo suporte disponível pelo {\it framework} DEAP para manipulação de algorítmos
genéticos, aumentando a rapidez no desenvolvimento do projeto, já que contém os 
operadores básicos dos GAs. É uma linguagem interpretada, multi-paradigma com 
alta legibilidade, sendo muito utilizada para o ensino.  

%%%%%%%%%%%%%%%%%%%%%%%%%%%%%%%%%%%%%%%%%%%%%%%%%%%%%%%%%%%%%%%%%%%%%%%%%%%%%%%%
%%%%%%%%%%%%%%%%%%%%%%%%%%%%%%%%%%%%%%%%%%%%%%%%%%%%%%%%%%%%%%%%%%%%%%%%%%%%%%%%
\subsection{Distributed Evolutionary Algorithms in Python (DEAP)}
O {\it framework} Distributed Evolutionary Algorithms in {\it Python} (DEAP) 
proporciona uma prototipação rápida de GAs, com estruturas de dados transparentes
e algorítmos explícitos \cite{deap}.

Seus módulos disponibilizam um grande aparato de ferramentas para a construção de
um algorítmo genético. Fazendo com que o programador se preocupe mais com a 
modelagem do problema.
%%%%%%%%%%%%%%%%%%%%%%%%%%%%%%%%%%%%%%%%%%%%%%%%%%%%%%%%%%%%%%%%%%%%%%%%%%%%%%%%
%%%%%%%%%%%%%%%%%%%%%%%%%%%%%%%%%%%%%%%%%%%%%%%%%%%%%%%%%%%%%%%%%%%%%%%%%%%%%%%%
\subsection{scikit-learn}
Scikit-learn é uma biblioteca de código aberto para programação em linguagem 
{\it Python} \cite{scikit_learn}. Faz uso de outras bibliotecas matemáticas e 
científicas (por exemplo {\it NumPy} e {\it SciPy}). Dá suporte para clusterização,
classificação e regressão, o que a torna interessante para aplicação nesse projeto.

%% Otimização de funções     %%%%%%%%%%%%%%%%%%%%%%%%%%%%%%%%%%%%%%%%%%%%%%%%%%%%
%%%%%%%%%%%%%%%%%%%%%%%%%%%%%%%%%%%%%%%%%%%%%%%%%%%%%%%%%%%%%%%%%%%%%%%%%%%%%%%%%
% \section{Otimização de funções}
% O primeiro contato com os GAs foi feito com o problema de otimizar uma função em
% um dado intervalo, ou seja, encontrar o ponto de máximo utilizando o 
% {\it framework DEAP}. Para isso, o cromossomo foi representado como um valor 
% inteiro pra o eixo {\it x}, inicialmente, gerado de maneira psuedo-aleatória. 
% Esse valor inteiro foi convertido para sua forma binária. Para o cruzamento foi 
% escolhido o {\it one point crossover}.

% A função {\it fitness} foi definida como 
% \begin{equation} fitness(x) = 100 - x^2 \end{equation},
% ou seja, o melhor indivíduo será o cromossomo {\it x} que maximizar a função. De
% maneira análoga se resolve o problema de minimizar uma função. 

% Ao executar o algorítmo com chance de {\it crossover} de 75\%, chance de mutação
% de 5\% e 20 gerações o resultado convergiu rapidamente para o esperado 
% \begin{equation} x = 0 \end{equation}.