%%%%%%%%%%%%%%%%%%%%%%%%%%%%%%%%%%%%%%%%%%%%%%%%%%%%%%%%%%%%%%%%%%%%%%%%%%%%%%%%%%%%%%%%
%%%%%%%%%%%%%%%%%%%%%%%%              Metodologia              %%%%%%%%%%%&%%%%%%%%%%%%%
%%%%%%%%%%%%%%%%%%%%%%%%%%%%%%%%%%%%%%%%%%%%%%%%%%%%%%%%%%%%%%%%%%%%%%%%%%%%%%%%%%%%%%%%
Dada a dificuldade de se trabalhar com grandes bancos de dados (este caso, por exemplo), é importante que o máximo de informações relevantes possam ser extraídas com confiabilidade dentro de um espaço de tempo plausível. O objetivo dessa fase inicial é criar um primeiro modelo híbrido de {\it clustering} com algorítmos genéticos que atenda a esses princípios.

No primeiro momento, foram estudados os conceitos e algorítmos fundamentais de GAs. Em seguida foi estudado e implementado o algorítmo clássico {\it K-means}\footnote[1]{\url{https://github.com/IcaroMarcelino/Testes_Deap}} seguindo a descrição do algorítmo clássico encontrado na literatura \cite{data_clustering}, entendendo seu funcionamento e como seria possível melhorá-lo usando GAs. A última etapa consiste em efetivamente elaborar o algorítmo híbrido, que é descrito na próxima seção.

Espera-se encontrar, comparando o algorítmo clássico e o híbrido, resultados mais consistentes do segundo se ambos forem executados apenas uma vez. Apesar da execução mais rápida do primeiro, para resultados confiáveis são necessárias diversas execuções. 

Para comprovar a hipótese, será utilizado o banco de dados iris, para se ter resultados mais rápidos.